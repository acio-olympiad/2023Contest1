\documentclass{article}
\usepackage{multicol}
\usepackage[utf8]{inputenc}
\usepackage[textwidth=460pt, voffset=0pt]{geometry}
\usepackage{fancyhdr}
\begin{document}
\title{\vspace{-5ex}Ice-cream Obsession 2}
\author{\vspace{-5ex}}
\date{\vspace{-5ex}}
\pagestyle{fancy}
\fancyhf{}
\lhead{ACIO 2023 Contest 1}
\rfoot{Page \thepage}

\begin{center}
\huge{Ice-cream Obsession 2}\small\\
\vspace{5ex}
\begin{tabular}{|c|c|} 
\hline
Time Limit & Memory Limit \\
\hline
1 second & 128 MB \\

\hline
\end{tabular}
\end{center}
\section*{Statement}

After Tommy tells you in extensive detail how and why he likes ice cream so much, you have been thoroughly convinced to sign up to as many ice-cream tasting events in your city as possible.

Your country has $N$ cities (numbered 1 to $N$) and $M$ bidirectional roads between them of varying length. Over the course of the next $P$ days, there will be an ice cream tasting event on each day. The $i$th such event is located at city $a_i$. Note that $M \le N + 10$.

You plan to be at every single ice cream tasting event, by starting in city $a_1$, travelling to city $a_2$, and so on, until you end up at city $a_P$. Since this is quite a far distance, you want to determine the minimum total distance you need to travel.

\section*{Input}

\begin{itemize}
\item The first line of input contains the integers $N$, $M$ and $P$.
\item The next $M$ lines contain three integers: $u$, $v$, $w$, meaning there is a bidirectional road connecting cities $u$ and $v$ of length $w$.
\item The next line contains $P$ integers: $a_1\ \dots \ a_P$
\end{itemize}

\section*{Output}

Output a single integer; the shortest total path length. As this number can be quite large, make sure you store your answer as a 64-bit integer.

\begin{multicols}{2}
\section*{Sample Input 1}
{\tt
5 5 7\\
1 2 1\\
1 3 2\\
2 4 2\\
3 4 3\\
4 5 1\\
3 1 4 1 5 4 3
}
\columnbreak
\section*{Sample Output 1}
{\tt
16
}
\end{multicols}

\newpage
\section*{Constraints}
\begin{itemize}
\item $1\leq N, M \leq 10^5$
\item $1 \le P \le 10^5$
\item $1 \le w \le 10^4$
\item $M \le N + 10$
\item $1 \le a_i \le N$
\item The graph has no self-edges.
\item There exists at least one path between any two cities.
\end{itemize}

\section*{Subtasks}
\begin{tabular}{l*{6}{c}r}
Number & Points & Other constraints\\
\hline
1 & 15 & $N, P \le 1000$ \\
2 & 25 & $M = N - 1$. That is, the cities form a tree.\\
3 & 25 & $M \le N$ \\
4 & 35 & No other constraints
\end{tabular}
\end{document}
